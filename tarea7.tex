\documentclass{article}

% formato
\usepackage[margin = 1.5cm, letterpaper]{geometry}
\usepackage[utf8]{inputenc}

%tablas
\usepackage{graphicx}

%formato ecuaciones
\usepackage{amsmath}

% símbolos
\usepackage{amssymb}

% manejo de tablas
\usepackage{float}

\begin{document}
    \title{
        Autómatas y Lenguajes formales \\
        Ejercicio Semanal 7
    }

    \author{
        Sandra del Mar Soto Corderi \\
        Edgar Quiroz Castañeda
    }

    \date{
        22 de marzo del 2019
    }
    
    \maketitle

    \begin{enumerate}
        \item {
            Para cada 	ANFD, resuelve los siguientes incisos.
            \begin{enumerate}
                \item Construye un autómata mínimo equivalente, mostrando paso 
                a paso el proceso de construcción.
                
                \item Da una expresión regular $\alpha$ correspondiente al 
                lenguaje aceptado por el autómata usando el método de ecuaciones
                características.
            \end{enumerate}
        }
       \end{enumerate}
    	\begin{enumerate}
    		\item {
    			Autómata 1
    			\begin{figure} [H]
    				\centering
    				\includegraphics[scale=.60]{automata1Tarea7.png}
    				\caption{El autómata M}
    			\end{figure}
    			 \begin{enumerate}
    				\item {
    				Autómata mínimo\\
    				Para minimizar el autómata, primero quitamos los estados inaccesibles, podemos ver que no hay manera de acceder a $q_{4}$, así que lo eliminamos del autómata y eliminamos las transiciones hacia el.\\
    				
    				Comencemos con la partición inducida por $[\equiv_{0}] = 
    				\{A = F, B = Q \setminus F\}$.\\
    				Evaluando $\delta$ para obtener las clases de $\equiv_{1}$
    				
    				\begin{table}[H]
    					\centering
    					\begin{tabular}{|l|l|l|l|l|l|l|}
    						\hline
    						A & $q_{1}$ & $q_{2}$ & $q_{3}$ & $q_{5}$ & $q_{6}$ & $q_{7}$ \\ \hline
    						a & A       & B       & B       & A       & B       & B       \\ \hline
    						b & A       & B       & A       & A       & B       & A       \\ \hline
    					\end{tabular}
    					\quad
    					\begin{tabular}{|l|l|l|}
    						\hline
    						B &  $q_{0}$ &  $q_{8}$ \\ \hline
    						a &  B &  A       \\ \hline
    						b &  B &  A      \\ \hline
    					\end{tabular}
    				\end{table}
    				
    				De $A$ se refinan tres nuevas clases , y de $B$ se refinan dos nuevas clases.\\
    				Por lo que $[\equiv_{1}] = \{B=\{q_{0}\}, F=\{q_{8}\} , C = \{q_{1}, q_{5}\},
    				D = \{q_{2}, q_{6}\}, E = \{q_{3}, q_{7}\}\}$.\\
    				Hay que evaluar $\delta$ en 
    				los elementos de $C, D, E$ para obtener las clases de $\equiv_{2}$.\\
    				\begin{table}[H]
    					\centering
    					\begin{tabular}{|l|l|l|}
    						\hline
    						C & $q_{1}$ & $q_{5}$ \\ \hline
    						a & D       & D       \\ \hline
    						b & E       & E       \\ \hline
    					\end{tabular}
    				    \quad
    				    \begin{tabular}{|l|l|l|}
    				    	\hline
    				    	D & $q_{2}$ & $q_{6}$ \\ \hline
    				    	a & G       & G       \\ \hline
    				    	b & G       & G       \\ \hline
    				    \end{tabular}
    			    	\quad
    			    	\begin{tabular}{|l|l|l|}
    			    		\hline
    			    		E & $q_{3}$ & $q_{7}$ \\ \hline
    			    		a & G       & G     \\ \hline
    			    		b & E       & E       \\ \hline
    			    	\end{tabular}
    				\end{table} 
    				Entonces $[\equiv_{2}] = \{C, D, E \}$.
    				
    				No se genera ningún refinamiento, por lo que el 
    				proceso ya ha acabado.\\
    				El autómata mínimo entonces es:
    					\begin{figure} [H]
    					\centering
    					\includegraphics[scale=.60]{automata1,2Tarea7.png}
    					\caption{El autómata mínimo M'}
    				\end{figure}
    			}
    		
    			\item {
    			Expresión regular usando el sistema de ecuaciones característico:\\
    			
    			Consideremos: $B = 0, C = 1, D = 2, E = 3, F= 4$\\ Definimos el sistema de ecuaciones a resolver:\\
    			\begin{align*}
    				L_{0} &= aL_{1} + bL_{1}\\
    				L_{1} &= aL_{2} + bL_{3} + \epsilon\\
    				L_{2} &= aL_{4} + bL_{4} + \epsilon\\
    				L_{3} &= aL_{4} + bL_{3} + \epsilon\\
    				L_{4} &= aL_{4} + bL_{4}
    			\end{align*}
    			Simplificamos algunas de las ecuaciones
    			\begin{align*}
    				L_{0} &= (a+b)L_{1}\\
    				L_{2} &= (a+b)L_{4} + \epsilon\\
    				L_{4} &= (a+b)L_{4}
    			\end{align*}
    			Aplicamos el lema de Arden en $L_{4}$
    			\begin{align*}
					L_{4} &= (a+b)L_{4}\\
					L_{4} &= (a+b)L_{4} + \varnothing \\
					L_{4} &= (a+b)(a+b)^*\varnothing + \varnothing \\
					L_{4} &= \varnothing
    			\end{align*}
    			Sustituimos $L_{4}$ en $L_{3}$ aplicamos el lema de Arden en $L_{3}$
    			\begin{align*}
    				L_{3} &= aL_{4} + bL_{3} + \epsilon\\
					L_{3} &= a\varnothing + bL_{3} + \epsilon\\
					L_{3} &= \varnothing + bL_{3} + \epsilon\\
					L_{3} &= bL_{3} + \epsilon\\
					L_{3} &= b(b^*\epsilon) + \epsilon\\
					L_{3} &= bb^* + \epsilon\\
					L_{3} &= b^*
    			\end{align*}
    			Sustituimos $L_{4}$ en $L_{2}$
    			\begin{align*}
    				L_{2} &= (a+b)L_{4} + \epsilon\\
    				L_{2} &= (a+b)\varnothing + \epsilon\\
    				L_{2} &= \varnothing + \epsilon \\
    				L_{2} &= \epsilon
    			\end{align*}
    			Sustituimos $L_{2}$ y $L_{3}$ en $L_{1}$
    			\begin{align*}
    				L_{1} &= aL_{2} + bL_{3} + \epsilon\\
    				L_{1} &= a(\epsilon) + b(b^*) + \epsilon\\
    				L_{1} &= a + (bb^* + \epsilon)\\
    				L_{1} &= a + b^*
    			\end{align*}
    			Sustituimos $L_{1}$ en $L_{0}$
    			\begin{align*}
    				L_{0} &= (a+b)L_{1}\\
    				L_{0} &= (a+b)(a + b^*)
    			\end{align*}
    			Por lo tanto, la expresión regular correspondiente al lenguaje aceptado es: $\alpha = (a+b)(a + b^*)$
    		}
    			\end{enumerate}
			}
    	\item {
    		Autómata 2
    		\begin{figure} [H]
    			\centering
    			\includegraphics[scale=.60]{automata2Tarea7.png}
    			\caption{El autómata A}
    		\end{figure}
    
            \begin{enumerate}
                \item {
                Autómata mínimo\\
                Comencemos con la partición inducida por $[\equiv_{0}] = 
                \{A = F, B = Q \setminus F\}$.\\
                Evaluando $\delta$ para obtener las clases de $\equiv_{1}$
                
                    \begin{table}[H]
                        \centering
                        \begin{tabular}{|l|l|l|}
                            \hline
                            $A$ & $q_{2}$ & $q_4$ \\ \hline
                            $a$      & $B$     & $B$   \\ \hline
                            $b$      & $B$     & $B$   \\ \hline
                        \end{tabular}
                        \quad
                        \begin{tabular}{|l|l|l|l|l|l|}
                            \hline
                            $B$ & $q_{0}$ & $q_{1}$ & $q_{3}$ & $q_{5}$ & $q_{6}$ \\ \hline
                            $a$      & $B$     & $B$     & $A$     & $B$     & $B$     \\ \hline
                            $b$      & $B$     & $A$     & $B$     & $B$     & $B$     \\ \hline
                        \end{tabular}
                    \end{table}

                $A$ no cambia, y de $B$ se refinan tres nuevas clases.\\
                Por lo que $[\equiv_{1}] = \{A, C = \{q_{0}, q_{5}, q_{6}\},
                D = \{q_{1}\}, E = \{q_{3}\}\}$.\\
                Cómo $D$ y $E$ son unitarios, sólo hay que evaluar $\delta$ en 
                los elementos de $C$ para obtener las clases de $\equiv_{2}$.\\
                \begin{table}[H]
                    \centering
                    \begin{tabular}{|l|l|l|l|}
                    \hline
                    $C$ & $q_{0}$ & $q_{5}$ & $q_{6}$ \\ \hline
                    $a$ & $D$     & $C$     & $C$     \\ \hline
                    $b$ & $E$     & $C$     & $C$     \\ \hline
                    \end{tabular}
                \end{table} 
                Entonces $[\equiv_{1}] = \{A, D, E, G = \{q_{0}\}, H = \{q_{5},
                q_{6}\}\}$.
                Evaluando $\delta$ sobre $H$ para obtener las clases de 
                $\equiv_{3}$
                \begin{table}[H]
                    \centering
                    \begin{tabular}{|l|l|l|l|}
                    \hline
                    $H$ & $q_{5}$ & $q_{6}$ \\ \hline
                    $a$ & $H$     & $H$     \\ \hline
                    $b$ & $H$     & $H$     \\ \hline
                    \end{tabular}
                \end{table}
                Por lo que no se generó ningún refinanmiento, por lo que el 
                proceso ya ha acabado.\\
                El autómata mínimo entonces es 
                \begin{figure} [H]
                	\centering
                	\includegraphics[scale=.60]{automata2,2Tarea7.png}
                	\caption{El autómata mínimo A'}
                \end{figure}
            }
        	\item{
        	}
            \end{enumerate}
    	}
    \end{enumerate}
\end{document}