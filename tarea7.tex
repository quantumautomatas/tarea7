\documentclass{article}

% formato
\usepackage[margin = 1.5cm, letterpaper]{geometry}
\usepackage[utf8]{inputenc}

%tablas
\usepackage{graphicx}

%formato ecuaciones
\usepackage{amsmath}

% símbolos
\usepackage{amssymb}

% manejo de tablas
\usepackage{float}

\begin{document}
    \title{
        Autómatas y Lenguajes formales \\
        Ejercicio Semanal 7
    }

    \author{
        Sandra del Mar Soto Corderi \\
        Edgar Quiroz Castañeda
    }

    \date{
        21 de marzo del 2019
    }
    
    \maketitle

    \begin{enumerate}
        \item {
            Para cada 	ANFD, resuelve los siguientes incisos.
            \begin{enumerate}
                \item Construye un autómata mínimo equivalente, mostrando paso 
                a paso el proceso de construcción.
                
                \item Da una expresión regular $\alpha$ correspondiente al 
                lenguaje aceptado por el autómata usando el método de ecuaciones
                características.
            \end{enumerate}
        }
       \end{enumerate}
    	\begin{enumerate}
    		\item {
    			Autómata 1
    			\begin{figure} [H]
    				\centering
    				\includegraphics[scale=.60]{automata1Tarea7.png}
    				\caption{El autómata M}
    			\end{figure}
				}
    	
    	\item {
    		Autómata 2
    		\begin{figure} [H]
    			\centering
    			\includegraphics[scale=.60]{automata2Tarea7.png}
    			\caption{El autómata A}
    		\end{figure}
    
            \begin{enumerate}
                \item Autómata mínimo\\
                Comencemos con la partición inducida por $[\equiv_{0}] = 
                \{A = F, B = Q \setminus F\}$.\\
                Evaluando $\delta$ para obtener las clases de $\equiv_{1}$
                
                    \begin{table}[H]
                        \centering
                        \begin{tabular}{|l|l|l|}
                            \hline
                            $A$ & $q_{2}$ & $q_4$ \\ \hline
                            $a$      & $B$     & $B$   \\ \hline
                            $b$      & $B$     & $B$   \\ \hline
                        \end{tabular}
                        \quad
                        \begin{tabular}{|l|l|l|l|l|l|}
                            \hline
                            $B$ & $q_{0}$ & $q_{1}$ & $q_{3}$ & $q_{5}$ & $q_{6}$ \\ \hline
                            $a$      & $B$     & $B$     & $A$     & $B$     & $B$     \\ \hline
                            $b$      & $B$     & $A$     & $B$     & $B$     & $B$     \\ \hline
                        \end{tabular}
                    \end{table}

                $A$ no cambia, y de $B$ se refinan tres nuevas clases.\\
                Por lo que $[\equiv_{1}] = \{A, C = \{q_{0}, q_{5}, q_{6}\},
                D = \{q_{1}\}, E = \{q_{3}\}\}$.\\
                Cómo $D$ y $E$ son unitarios, sólo hay que evaluar $\delta$ en 
                los elementos de $C$ para obtener las clases de $\equiv_{2}$.\\
                \begin{table}[H]
                    \centering
                    \begin{tabular}{|l|l|l|l|}
                    \hline
                    $C$ & $q_{0}$ & $q_{5}$ & $q_{6}$ \\ \hline
                    $a$ & $D$     & $C$     & $C$     \\ \hline
                    $b$ & $E$     & $C$     & $C$     \\ \hline
                    \end{tabular}
                \end{table} 
                Entonces $[\equiv_{1}] = \{A, D, E, G = \{q_{0}\}, H = \{q_{5},
                q_{6}\}\}$.
                Evaluando $\delta$ sobre $H$ para obtener las clases de 
                $\equiv_{3}$
                \begin{table}[H]
                    \centering
                    \begin{tabular}{|l|l|l|l|}
                    \hline
                    $H$ & $q_{5}$ & $q_{6}$ \\ \hline
                    $a$ & $H$     & $H$     \\ \hline
                    $b$ & $H$     & $H$     \\ \hline
                    \end{tabular}
                \end{table}
                Por lo que no se generó ningún refinanmiento, por lo que el 
                proceso ya ha acabado.\\
                El autómata mínimo entonces es 
            \end{enumerate}
    	}
    \end{enumerate}
\end{document}