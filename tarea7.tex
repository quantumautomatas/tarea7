\documentclass{article}

% formato
\usepackage[margin = 1.5cm, letterpaper]{geometry}
\usepackage[utf8]{inputenc}

%tablas
\usepackage{graphicx}

%formato ecuaciones
\usepackage{amsmath}

% símbolos
\usepackage{amssymb}

% manejo de tablas
\usepackage{float}

\begin{document}
    \title{
        Autómatas y Lenguajes formales \\
        Ejercicio Semanal 7
    }

    \author{
        Sandra del Mar Soto Corderi \\
        Edgar Quiroz Castañeda
    }

    \date{
        21 de marzo del 2019
    }
    
    \maketitle

    \begin{enumerate}
        \item {
            Para cada 	ANFD, resuelve los siguientes incisos.
            \begin{enumerate}
                \item Construye un autómata mínimo equivalente, mostrando paso 
                a paso el proceso de construcción.
                
                \item Da una expresión regular $\alpha$ correspondiente al 
                lenguaje aceptado por el autómata usando el método de ecuaciones
                características.
            \end{enumerate}
        }
       \end{enumerate}
    	\begin{enumerate}
    		\item {
    			Autómata 1
    			\begin{figure} [H]
    				\centering
    				\includegraphics[scale=.60]{automata1Tarea7.png}
    				\caption{El autómata M}
    			\end{figure}
				}
    	
    	\item {
    		Autómata 2
    		\begin{figure} [H]
    			\centering
    			\includegraphics[scale=.60]{automata2Tarea7.png}
    			\caption{El autómata A}
    		\end{figure}
    

    	}
    \end{enumerate}
\end{document}